
\setcounter{exo}{0}
\setcounter{que}{0}
\begin{minipage}{0.15\textwidth}
    Nom : 
\end{minipage}
\begin{minipage}{0.8\textwidth}
    \begin{flushright}
        Interrogation rapide - 2nde 1 - 18 novembre 2022
    \end{flushright}
\end{minipage}
\newline
\vspace{1pt}
\hrulefill 
\vspace{3pt}

\begin{exo}[De la représentation graphique au tableau de valeurs]

La courbe ci-dessous est la représentation de la fonction f dans un repère orthonormé 
\begin{center}
    \begin{tikzpicture}
        \begin{axis}[
        axis lines=middle,
        grid=major,
        xmin=-7.1,
        xmax=7.1,
        ymin=-2.1,
        ymax=7.1,
        xlabel=$x$,
        ylabel=$y$,
        xtick={-7,-6,...,7},
        ytick={-7,-6,...,7},
        scale=2.2,
        transform shape,
        ticklabel style={
                    fill=white
                },
        tick style={very thick},
        axis equal image,
        legend style={
        at={(rel axis cs:0,1)},
        anchor=north west,draw=none,inner sep=0pt,fill=gray!10}
        ]
        \addplot[color=red] coordinates {
            {w_coords}
            };
        \end{axis}
    \end{tikzpicture}
\end{center}

Complétez le tableau de valeurs suivant correspondant au graphique :
\newline

{\centering
        \renewcommand{\arraystretch}{2}
        \begin{tabular}{|c|*{7}{P{1cm}|}}
             \hline
             $x$& {w_xcoords} \\
             \hline
             $f(x)$& & & & & & & \\
             \hline
        \end{tabular}%
        
}
\end{exo}
\vspace{0.5cm}
\begin{exo}
    Répondez par Vrai ou Faux aux affirmations suivantes 
    \begin{enumerate}
        {w_questions}
    \end{enumerate}
\end{exo}

\begin{que}[Préparation du nouveau plan de classe]
Répondez aux questions suivantes:
    \begin{enumerate}
        \item Citez des élèves à côté desquels \textbf{vous} vous sentez efficace au travail
        \begin{multicols}{2}
            \begin{itemize}
                \item ...............
                \item ...............
                \item ...............
                \item ...............
            \end{itemize}
        \end{multicols}
        \item Citez des élèves à côté desquels \textbf{vous} ne vous sentez pas efficace au travail
        \begin{multicols}{2}
            \begin{itemize}
                \item ...............
                \item ...............
                \item ...............
                \item ...............
            \end{itemize}
        \end{multicols}
        \item Avez-vous une contrainte ou un désir sur votre position dans la classe ?
    \end{enumerate}
\end{que}

\fbox{
    \begin{minipage}[t][1.8cm][t]{0.965\linewidth}
        ~
    \end{minipage}
}


