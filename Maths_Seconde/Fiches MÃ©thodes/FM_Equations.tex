\documentclass[10pt,a4paper]{book}

\usepackage[utf8]{inputenc}
\usepackage[T1]{fontenc}
\usepackage[french]{babel}
%\renewcommand\sfdefault{phv}
\usepackage[left=2cm, right=2cm, bottom=1.85cm, top=1.5cm]{geometry}

\usepackage{fancyhdr}
\usepackage{framed}
\pagestyle{fancy}
\usepackage{hyperref}
%\renewcommand{\headrulewidth}{1pt}
%\fancyhead[C]{\textbf{page \thepage}} 
\fancyhead[L]{Fiche méthode : Equation}
\fancyhead[R]{Seconde}

\renewcommand{\footrulewidth}{1pt}
%\fancyfoot[C]{\textbf{page \thepage}} 
\fancyfoot[L]{Lycée Jacques Brel}
\fancyfoot[R]{Année 2022-2023}

%%%%%%%%%%%%%%%%%%%%%%%%%%%%%%%%%%%%%%%%%%%%%%%%%%

%\newcommand{\TDoc}[1]{
%\begin{center}
%{\setlength{\fboxsep}{10pt}  % Ecart texte-boite
%\shadowbox{\textbf{\Large{#1}}}}
%\end{center}
%\vspace{1.5cm}}
    
 \usepackage{fancybox}   %pour l'encadré du titre shadowbox
 \usepackage{stmaryrd}   %pour utiliser correctement les crochets pour les ensembles de définitions
 \usepackage[normalem]{ulem}
 \usepackage{graphicx}
 %\usepackage{wrapfig} %texte coulé autour d'une image
 \usepackage{soul} % souligné
 \usepackage{nonfloat}
 \usepackage[standard]{ntheorem}
 \usepackage{array}
 \usepackage{arydshln}
 \usepackage{graphicx}
%MATHEMATIQUES
\usepackage{amsmath,amsfonts} 

\usepackage{tikz}
\newcommand{\N}{\mathbb{N}}
\newcommand{\Z}{\mathbb{Z}}
\newcommand{\D}{\mathbb{D}}
\newcommand{\Q}{\mathbb{Q}}
\newcommand{\R}{\mathbb{R}}
\newcommand{\Co}{\mathbb{C}}
\newcommand{\K}{\mathbb{K}}
\newcommand{\F}{\mathbb{F}}


%%%%%%%%%%%%%%%%%%%%%%%%%%%%%%%%%%%

\makeatletter
%%%%%%%%%%%%%%%%%%% debut fichier boiboites.sty %%%%%%%%%%%%%%%%%%%%%%
\RequirePackage{xkeyval}
\RequirePackage{tikz}
\usetikzlibrary{intersections}
\usetikzlibrary{positioning}
\usetikzlibrary{3d}
\RequirePackage{amssymb}

\define@key{boxedtheorem}{titlecolor}{\def\titlecolor{#1}}
\define@key{boxedtheorem}{titlebackground}{\def\titlebackground{#1}}
\define@key{boxedtheorem}{background}{\def\background{#1}}
\define@key{boxedtheorem}{titleboxcolor}{\def\titleboxcolor{#1}}
\define@key{boxedtheorem}{boxcolor}{\def\boxcolor{#1}}
\define@key{boxedtheorem}{thcounter}{\def\thcounter{#1}}
\define@key{boxedtheorem}{size}{\def\size{#1}}
\presetkeys{boxedtheorem}{titlecolor = black, titlebackground = white, background = white,%
                         titleboxcolor = black, boxcolor = black, thcounter=, size = .9\textwidth}{}

\newcommand{\couleurs}[1][]{%
    \setkeys{boxedtheorem}{#1}
    \tikzstyle{fancytitle} =[draw=\titleboxcolor, rounded corners, fill=\titlebackground,
                            text= \titlecolor]
    \tikzstyle{mybox} = [draw=\boxcolor, fill=\background, very thick,
                        rectangle, rounded corners, inner sep=10pt, inner ysep=20pt]
}


%Commande generique pour faire un joli encadre
\newsavebox{\boiboite}
\newcommand{\titre}{Titre}
\newenvironment{boite}[2][]%
    {%
    \renewcommand{\titre}{#2}
    \couleurs[#1]
    \begin{lrbox}{\boiboite}%
     \begin{minipage}[!h]{\size}
    }%
    {%
     \end{minipage}
    \end{lrbox}
    \begin{center}
    \begin{tikzpicture}
    \node [mybox] (box){\usebox{\boiboite}};
    \node[fancytitle, right=10pt] at (box.north west) {\titre};
    \end{tikzpicture}
    \end{center}
    }

\newcommand{\newboxedtheorem}[4][]{%
    \couleurs[#1]
    \@ifnotempty{#4}{%
      \@ifundefined{the#4}{\@ifundefined{\thcounter}{\newcounter{#4}}{%
      \newcounter{#4}[\thcounter ] } } { }%
    }
    \newenvironment{#2}[1][]{%
    \@ifnotempty{#4}{\refstepcounter{#4}}
    \begin{boite}[#1]{\textbf{#3\@ifnotempty{#4}{ \csname the#4\endcsname}}\@ifnotempty{##1}{
    (##1)}}
    }%
    {%
    \end{boite}
    }
}

\newcommand{\newboxedtheoreme}[4][]{%
    \couleurs[#1]
    \@ifnotempty{#4}{%
      \@ifundefined{the#4}{\@ifundefined{}{}{%
      } } { }%
    }
    \newenvironment{#2}[1][]{%
    \@ifnotempty{#4}{\refstepcounter{#4}}
    \begin{boite}[#1]{\textbf{#3\@ifnotempty{#4}{ \csname the#4\endcsname}}\@ifnotempty{##1}{
    (##1)}}
    }%
    {%
    \end{boite}
    }
}
%%%%%%%%%%%%%%%%%%%% end fichier boiboites.sty %%%%%%%%%%%%%%%%%%%%%%
\makeatother
\newboxedtheorem{theo}{Théorème}{theorem}
\newboxedtheorem{de}{D\'efinition}{theorem}
\newboxedtheorem{prop}{Propriété}{theorem}
\newboxedtheorem{pro}{Proposition}{theorem}
\newboxedtheorem{ton}{Notation}{theorem}
\newtheorem{exo}{Exercice}
\newboxedtheorem{exe}{Exemple}{theorem}
\newboxedtheorem{met}{Méthode}{theorem}
\newboxedtheorem{cor}{Corolaire}{theorem}
\newboxedtheoreme{conc}{Conclusion}{theorem*}
\newboxedtheoreme{demo}{\textbf{Démonstration guidée}}{theorem}
%%%%%%%%%%%%%%%%%%%%%%%%%%%%%%
%%%%%%%%%%%%%%%%%%%%%%%%%%%%%
\newlength{\longA}
\newlength{\longB}
\newenvironment{BoiteShadow}[3][\linewidth]{%
\addtolength{\longA}{#2}
\addtolength{\longB}{#3}
\begin{Sbox}\begin{minipage}{#1}}%
{\end{minipage}\end{Sbox}%
\setlength{\fboxsep}{\longA}
\setlength{\shadowsize}{\longB}
\shadowbox{\unhbox\@Sbox}\par}
\makeatother
%\author{Augustin WENGER}

%\title{Cours de Terminale STMG 2 2022-2023}
%\date{}

%%%% fin du préambule, on passe au contenu : tout le texte entre
%%%% \begin{document} et \end{document} 
\pagenumbering{gobble} % removes page numbering
\begin{document}
%\chapter{Suites arithmétiques et géométriques}

\chapter{Fiche méthode : Equations}

\section{Vocabulaire}

\begin{de}
    Une \underline{équation à une inconnue} est une  \textbf{égalité} entre deux expressions, chacune étant constituée de nombres et d'au plus une inconnue, que l'on symbolise par une lettre (le plus souvent $x$).
    
\end{de}

Exemples d'équations:

$x = 4$

$4 = 8$

$x = x^2$



\begin{de}
    Un nombre est dit \underline{solution} d'une équation si remplacer l'inconnue par ce nombre rend l'égalité vraie
\end{de}

Remarque : Une équation peut avoir zéro, une, plusieurs, ou même une infinité de solutions.

Exemples d'équation à une inconnue :
    \begin{enumerate}
        \item N'admettant aucune solution :
        \item Ayant une et une seule solution :
        \item Ayant plusieurs solutions :
        \item Ayant un nombre infini de solutions :
        
    \end{enumerate}



\begin{exo} 
Les équations apparaissent naturellement quand on essaye de résoudre des problèmes concrets. Sans les résoudre, exprimer des équations pour résoudre les différents problèmes, en précisant comment on définit l'inconnue.
    \begin{enumerate}
        \item Trois adultes et trente enfants se rendent à une sortie scolaire au théâtre. Une place enfant vaut 7 euros de moins qu'une place adulte. Le montant total payé par la classe est de 615 euros. Combien coûte une place adulte ?
        \item Quelle est la longueur du côté d'un carré dont l'aire, en centimètres carrés, est égale à son périmètre en centimètres ?
        \item Quel nombre augmenté de 1 est égal à son carré?
    \end{enumerate}
\end{exo}


\section{Travailler avec une équation}

La problématique générale en jeu est de déterminer \textbf{quelles valeurs numériques de l'inconnue rendent l'égalité vraie}.

Pour cela, on sera amené à faire deux choses principalement :

\begin{enumerate}
    \item Vérifier si une valeur donnée de l'inconnue est solution de l'équation
    \item Déterminer tous les nombres qui sont solutions de l'équation
    
    
\end{enumerate}


\subsection{Vérifier si une valeur donnée de l'inconnue est solution}



\begin{met} 
Pour vérifier si une valeur est solution de l'équation, on remplace l'inconnue par cette valeur, on effectue le calcul, et on vérifie si l'égalité est vérifiée.
\end{met}

\begin{exo} 
Vérifier si $x=2$ est solution des équations suivantes :
    \begin{enumerate}
        \item $-x - 2 = 0$
        \item $x^2 - 6x + 8 = 0$
    \end{enumerate}
\end{exo}

\begin{exo} 
Parmi les valeurs suivantes de $x$, lesquelles sont solutions de $x^3 - 7x^2 + 14x - 8 = 0$ ?
    \begin{enumerate}
        \item $x=1$
        \item $x=2$
        \item $x=3$
        \item $x=4$
    \end{enumerate}
\end{exo}



\subsection{Résoudre une équation}

\begin{de}
    \underline{Résoudre} une équation, c'est déterminer l'ensemble de ses solutions, c'est à dire \textbf{toutes} les valeurs de l'inconnue qui rendent l'égalité vraie.
\end{de}

Résoudre une équation est par définition beaucoup plus compliqué que de tester si une valeur donnée est solution. On ne sait résoudre que très peu d'équations.

Pour cela, on utilise la stratégie générale de résolution des problèmes de mathématiques : on modifie l'énoncé \textbf{sans en changer le sens} jusqu'à faire apparaître une équation que l'on sait résoudre.

\begin{prop}[Admise]
    On peut modifier une équation sans en changer le sens en effectuant l'une des opérations suivantes :
    \begin{enumerate}
        \item On ajoute (ou on retranche) le même nombre à gauche et à droite de l'égalité.
        \item On multiplie (ou on divise) par le même nombre \textbf{non nul} à gauche et à droite de l'égalité.
    \end{enumerate}
\end{prop}

Initialement, on ne sait résoudre que les équations de la forme $x=a$ (où $a$ ne dépend pas de $x$).  Par exemple, la seule valeur de $x$ qui rende vraie l'égalité $x=8$, c'est ...  $x=8$.

On veut donc, par une série d'opérations qui ne changent pas le sens de l'équation (c'est à dire qui conservent l'ensemble des solutions) obtenir une équation de cette forme.


\begin{exo} Transformer en utilisant seulement les deux opérations autorisées chaque équation suivante en une équation de la forme $x=a$ sans en changer le sens:
    \begin{enumerate}
        \item $x - 4 = 0$
        \item $-x + 2 = 0$
        \item $ 3x - 7 = 8$
        \item $ 5x + 2 = 3x + 6$
        \item $ \frac{1}{x+2} = 3$
    \end{enumerate}
\end{exo}


\begin{exo}[Préparation à la résolution générale] 
    Résoudre les équations suivantes :
    \begin{enumerate}
        \item $x+6 = 0$
        \item $3x+2 = 0$
        \item $5x-3 = 0$
        \item $-7x - 4 = 0$
        \item $-9x + 1 = 0$
    \end{enumerate}
    Que remarquez vous dans l'expression des solutions par rapport aux équations de départ?
\end{exo}


\begin{prop}
    Soit $a$ et $b$ deux nombres avec $a$ non nul, l'équation $ax+b=0$ est appelée \underline{\'Equation du premier degré}, et admet une unique solution : $x = -\frac{b}{a}$
\end{prop}


\begin{exo} 
    \begin{enumerate}
        \item Retrouver la solution de l'équation en utilisant la même technique qu'à l'exercice précédent.
        \item On peut également montrer cette solution avec une méthode plus classique, si l'on connaît la formule de la solution : Montrer que $-\frac{b}{a}$ est une solution de l'équation.
        \item Montrer que c'est la seule solution.
    \end{enumerate}
\end{exo}

On sait désormais résoudre toutes les équations du type $ax+b=0$. On peut donc à partir de maintenant se contenter de se ramener à une équation du premier degré.

\end{document}
