\documentclass[10pt,a4paper]{exam}

\usepackage[utf8]{inputenc}
\usepackage[T1]{fontenc}
\usepackage[french]{babel}
%\renewcommand\sfdefault{phv}
\usepackage[left=2cm, right=2cm, bottom=1.85cm, top=1.5cm]{geometry}

\usepackage{fancyhdr}
\usepackage{framed}
\pagestyle{fancy}
\usepackage{hyperref}
%\renewcommand{\headrulewidth}{1pt}
%\fancyhead[C]{\textbf{page \thepage}} 
\fancyhead[L]{Nom :}
\fancyhead[R]{2nde 1 - Interrogation rapide (10 minutes)}

\renewcommand{\footrulewidth}{1pt}
%\fancyfoot[C]{\textbf{page \thepage}} 
\fancyfoot[L]{Nom :}
\fancyfoot[R]{2nde 1 - Interrogation rapide (10 minutes)}

%%%%%%%%%%%%%%%%%%%%%%%%%%%%%%%%%%%%%%%%%%%%%%%%%%

%\newcommand{\TDoc}[1]{
%\begin{center}
%{\setlength{\fboxsep}{10pt}  % Ecart texte-boite
%\shadowbox{\textbf{\Large{#1}}}}
%\end{center}
%\vspace{1.5cm}}
    
 \usepackage{fancybox}   %pour l'encadré du titre shadowbox
 \usepackage{stmaryrd}   %pour utiliser correctement les crochets pour les ensembles de définitions
 \usepackage[normalem]{ulem}
 \usepackage{graphicx}
 %\usepackage{wrapfig} %texte coulé autour d'une image
 \usepackage{soul} % souligné
 \usepackage{nonfloat}
 \usepackage[standard]{ntheorem}
 \usepackage{array}
 \usepackage{arydshln}
 \usepackage{graphicx}
%MATHEMATIQUES
\usepackage{amsmath,amsfonts} 

\usepackage{tikz}
\newcommand{\N}{\mathbb{N}}
\newcommand{\Z}{\mathbb{Z}}
\newcommand{\D}{\mathbb{D}}
\newcommand{\Q}{\mathbb{Q}}
\newcommand{\R}{\mathbb{R}}
\newcommand{\Co}{\mathbb{C}}
\newcommand{\K}{\mathbb{K}}
\newcommand{\F}{\mathbb{F}}


%%%%%%%%%%%%%%%%%%%%%%%%%%%%%%%%%%%

\makeatletter
%%%%%%%%%%%%%%%%%%% debut fichier boiboites.sty %%%%%%%%%%%%%%%%%%%%%%
\RequirePackage{xkeyval}
\RequirePackage{tikz}
\usetikzlibrary{intersections}
\usetikzlibrary{positioning}
\usetikzlibrary{3d}
\RequirePackage{amssymb}

\define@key{boxedtheorem}{titlecolor}{\def\titlecolor{#1}}
\define@key{boxedtheorem}{titlebackground}{\def\titlebackground{#1}}
\define@key{boxedtheorem}{background}{\def\background{#1}}
\define@key{boxedtheorem}{titleboxcolor}{\def\titleboxcolor{#1}}
\define@key{boxedtheorem}{boxcolor}{\def\boxcolor{#1}}
\define@key{boxedtheorem}{thcounter}{\def\thcounter{#1}}
\define@key{boxedtheorem}{size}{\def\size{#1}}
\presetkeys{boxedtheorem}{titlecolor = black, titlebackground = white, background = white,%
                         titleboxcolor = black, boxcolor = black, thcounter=, size = .9\textwidth}{}

\newcommand{\couleurs}[1][]{%
    \setkeys{boxedtheorem}{#1}
    \tikzstyle{fancytitle} =[draw=\titleboxcolor, rounded corners, fill=\titlebackground,
                            text= \titlecolor]
    \tikzstyle{mybox} = [draw=\boxcolor, fill=\background, very thick,
                        rectangle, rounded corners, inner sep=10pt, inner ysep=20pt]
}


%Commande generique pour faire un joli encadre
\newsavebox{\boiboite}
\newcommand{\titre}{Titre}
\newenvironment{boite}[2][]%
    {%
    \renewcommand{\titre}{#2}
    \couleurs[#1]
    \begin{lrbox}{\boiboite}%
     \begin{minipage}[!h]{\size}
    }%
    {%
     \end{minipage}
    \end{lrbox}
    \begin{center}
    \begin{tikzpicture}
    \node [mybox] (box){\usebox{\boiboite}};
    \node[fancytitle, right=10pt] at (box.north west) {\titre};
    \end{tikzpicture}
    \end{center}
    }

\newcommand{\newboxedtheorem}[4][]{%
    \couleurs[#1]
    \@ifnotempty{#4}{%
      \@ifundefined{the#4}{\@ifundefined{\thcounter}{\newcounter{#4}}{%
      \newcounter{#4}[\thcounter ] } } { }%
    }
    \newenvironment{#2}[1][]{%
    \@ifnotempty{#4}{\refstepcounter{#4}}
    \begin{boite}[#1]{\textbf{#3\@ifnotempty{#4}{ \csname the#4\endcsname}}\@ifnotempty{##1}{
    (##1)}}
    }%
    {%
    \end{boite}
    }
}

\newcommand{\newboxedtheoreme}[4][]{%
    \couleurs[#1]
    \@ifnotempty{#4}{%
      \@ifundefined{the#4}{\@ifundefined{}{}{%
      } } { }%
    }
    \newenvironment{#2}[1][]{%
    \@ifnotempty{#4}{\refstepcounter{#4}}
    \begin{boite}[#1]{\textbf{#3\@ifnotempty{#4}{ \csname the#4\endcsname}}\@ifnotempty{##1}{
    (##1)}}
    }%
    {%
    \end{boite}
    }
}
%%%%%%%%%%%%%%%%%%%% end fichier boiboites.sty %%%%%%%%%%%%%%%%%%%%%%
\makeatother
\newboxedtheorem{theo}{Théorème}{theorem}
\newboxedtheorem{de}{D\'efinition}{theorem}
\newboxedtheorem{prop}{Propriété}{theorem}
\newboxedtheorem{pro}{Proposition}{theorem}
\newboxedtheorem{nota}{Notation}{theorem}
\newtheorem{exo}{Exercice}
\newboxedtheorem{exe}{Exemple}{theorem}
\newboxedtheorem{cor}{Corolaire}{theorem}
\newboxedtheoreme{conc}{Conclusion}{theorem*}
\newboxedtheoreme{demo}{\textbf{Démonstration guidée}}{theorem}
%%%%%%%%%%%%%%%%%%%%%%%%%%%%%%
%%%%%%%%%%%%%%%%%%%%%%%%%%%%%
\newlength{\longA}
\newlength{\longB}
\newenvironment{BoiteShadow}[3][\linewidth]{%
\addtolength{\longA}{#2}
\addtolength{\longB}{#3}
\begin{Sbox}\begin{minipage}{#1}}%
{\end{minipage}\end{Sbox}%
\setlength{\fboxsep}{\longA}
\setlength{\shadowsize}{\longB}
\shadowbox{\unhbox\@Sbox}\par}
\makeatother
%\author{Augustin WENGER}

\title{Cours de Seconde 1 2022-2023}
\date{}

%%%% fin du préambule, on passe au contenu : tout le texte entre
%%%% \begin{document} et \end{document} 

\begin{document}

Parmi ces nombres, lequel (ou lesquels) sont solutions de l'équation   $x^2 - x - 2 = 0$ ?


\begin{oneparcheckboxes}
   \choice $-2$
   \choice $-1$
   \choice $0$
   \choice $1$
   \choice $2$
\end{oneparcheckboxes}

\vspace{10 mm}
 
Quelle(s) équation(s) ont exactement les mêmes solutions que $3x+2=0$ ?

\begin{oneparcheckboxes}
   \choice $9x+6=0$
   \choice $3x = 2$
   \choice $x= \frac{2}{3}$
   \choice $-3x = 2$
\end{oneparcheckboxes}
 
\vspace{10 mm}

Parmi ces nombres, lequel (ou lesquels) sont solutions de l'équation   $8x-2=0$ ?


\begin{oneparcheckboxes}
   \choice $\frac{1}{4}$
   \choice $0.4$
   \choice $-0.4$
   \choice $0.25$
\end{oneparcheckboxes}
 
\vspace{10 mm}

Si a est un nombre non nul, et b un nombre quelconque, $ax+b=0$  admet une et une seule solution qui est :

\begin{oneparcheckboxes}
   \choice $\frac{a}{b}$
   \choice $-\frac{a}{b}$
   \choice $\frac{b}{a}$
   \choice $-\frac{b}{a}$
\end{oneparcheckboxes}
 
\vspace{10 mm}
Parmi ces nombres, lequel (ou lesquels) sont solutions de l'équation   $\frac{1}{x}=-3$ ?

\begin{oneparcheckboxes}
   \choice $\frac{1}{3}$
   \choice $-0.3$
   \choice $-\frac{1}{3}$
   \choice $-3$
\end{oneparcheckboxes}
 
\vspace{85 mm}

Parmi ces nombres, lequel (ou lesquels) sont solutions de l'équation   $x^2 + x - 2 = 0$ ?


\begin{oneparcheckboxes}
   \choice $-2$
   \choice $-1$
   \choice $0$
   \choice $1$
   \choice $2$
\end{oneparcheckboxes}

\vspace{10 mm}
 
Quelle(s) équation(s) ont exactement les mêmes solutions que $3x-2=0$ ?

\begin{oneparcheckboxes}
   \choice $9x+6=0$
   \choice $3x = 2$
   \choice $x= \frac{2}{3}$
   \choice $-3x = 2$
\end{oneparcheckboxes}
 
\vspace{10 mm}

Parmi ces nombres, lequel (ou lesquels) sont solutions de l'équation   $8x+2=0$ ?


\begin{oneparcheckboxes}
   \choice $-\frac{1}{4}$
   \choice $0.4$
   \choice $-0.4$
   \choice $-0.25$
\end{oneparcheckboxes}
 
\vspace{10 mm}

Si a est un nombre non nul, et b un nombre quelconque, $ax+b=0$  admet une et une seule solution qui est :

\begin{oneparcheckboxes}
   \choice $-\frac{b}{a}$
   \choice $\frac{b}{a}$
   \choice $-\frac{a}{b}$
   \choice $\frac{a}{b}$
\end{oneparcheckboxes}
 
\vspace{10 mm}
Parmi ces nombres, lequel (ou lesquels) sont solutions de l'équation   $\frac{1}{x}=\frac{10}{3}$ ?

\begin{oneparcheckboxes}
   \choice $\frac{1}{3}$
   \choice $0.3$
   \choice $-\frac{1}{3}$
   \choice $3$
\end{oneparcheckboxes}

 
 
\end{document}

