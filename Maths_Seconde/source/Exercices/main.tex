%%%%%MISE EN PAGE%%%%%%%
    

\documentclass[10pt,a4paper]{article}
\usepackage[utf8]{inputenc}
\usepackage[T1]{fontenc}
\usepackage[french]{babel}
\usepackage{multicol}
\usepackage[left=2cm, right=2cm, bottom=1.85cm, top=1.5cm]{geometry}

\usepackage{fancyhdr}
\pagestyle{fancy}

%\renewcommand{\headrulewidth}{1pt}
%\fancyhead[C]{\textbf{page \thepage}} 
\fancyhead[L]{Chapitre 1}
\fancyhead[R]{Seconde}

\renewcommand{\footrulewidth}{1pt}
\fancyfoot[C]{\textbf{page \thepage}} 
\fancyfoot[L]{Lycée Jacques Brel}
\fancyfoot[R]{Année 2022-2023}


%\newcommand{\TDoc}[1]{
%\begin{center}
%{\setlength{\fboxsep}{10pt}  % Ecart texte-boite
%\shadowbox{\textbf{\Large{#1}}}}
%\end{center}
%\vspace{1.5cm}}
    
 \usepackage{fancybox}   %pour l'encadré du titre shadowbox
 \usepackage{stmaryrd}   %pour utiliser correctement les crochets pour les ensembles de définitions
 \usepackage[normalem]{ulem}
 \usepackage{graphicx}
 %\usepackage{wrapfig} %texte coulé autour d'une image
 \usepackage{soul} % souligné
 \usepackage{nonfloat}
%MATHEMATIQUES
\usepackage{amsmath,amsfonts} 
\usepackage{tikz}
\newcommand{\N}{\mathbb{N}}
\newcommand{\Z}{\mathbb{Z}}
\newcommand{\D}{\mathbb{D}}
\newcommand{\Q}{\mathbb{Q}}
\newcommand{\R}{\mathbb{R}}
\newcommand{\C}{\mathbb{C}}
\newcommand{\K}{\mathbb{K}}
\newcommand{\F}{\mathbb{F}}


\newtheorem{de}{Définition} % les définitions et les théorèmes sont
\newtheorem{theo}{Théorème}    % numérotés par section
\newtheorem{prop}[theo]{Proposition} 
\newtheorem{pro}[theo]{Propriété} 
\newtheorem{exe}{Exemple} 
\newtheorem{exo}{Exercice}  
\newtheorem{qe}{Question de cours}
%%%%%%%%%%%%
\makeatletter
\newlength{\longA}
\newlength{\longB}
\newenvironment{BoiteShadow}[3][\linewidth]{%
\addtolength{\longA}{#2}
\addtolength{\longB}{#3}
\begin{Sbox}\begin{minipage}{#1}}%
{\end{minipage}\end{Sbox}%
\setlength{\fboxsep}{\longA}
\setlength{\shadowsize}{\longB}
\shadowbox{\unhbox\@Sbox}\par}
\makeatother
%%%%%%%

\title{}

%%%%%%%%%%%%%%%%
\begin{document}

\begin{BoiteShadow}{10pt}{8pt}
\Huge{Feuille d'exercice $n^{\circ}1$: Ensembles de nombres}
 \end{BoiteShadow}
%\section{Connaissance du cours}
%Vous avez cinq minutes pour répondre aux questions suivantes. \textbf{Dans votre cahier}, répondre aux questions suivantes \textbf{sans regarder le cours}. Si vous n'arrivez pas à répondre au bout des cinq minutes, répondez aux question en allant voir le cours. 

%\begin{enumerate}
%\item Je connais l'ensemble des nombres entiers naturels, des nombres entiers relatifs, des nombres décimaux et des nombres rationnels? Les décrire.  
%\item Comment sont caractérisés les nombres décimaux? 
%\item Comment additionner et multiplier des nombres rationnels?
%\item Comment multiplier ou diviser par une puissance de $10$?	
%\end{enumerate}

\section{Ensembles de nombres}

\begin{exo}
    Parmi les cinq ensembles vus en cours, dire quel est le plus petit qui contient chacun des nombres suivants :
    \begin{enumerate}
        \item $-345$
        \item $2,0756$
        \item $78$
        \item $\frac{5}{9}$
        \item $\frac{235}{5}$
        \item $2\pi$
        \item $\sqrt{36}$
        \item $\frac{2,97}{0,01}$
        \item $-5 \times 10^{-2}$
        \item $24,63 \times 10$
    \end{enumerate}
\end{exo}

\begin{exo}
  Ecrire les nombres décimaux suivants sous forme d'une fraction de la forme 
  $\cfrac{a}{10^{n}}$. Où $a\in\Z$ et $n\in\N$.
  \begin{enumerate}
    \item $a=3,3967$
    \item $b=0,0031415235$
    \item $c=25,75$
    \item $d=0,2192$
  \end{enumerate}
\end{exo}

\begin{exo}
  Montrer que les nombres suivants sont décimaux, puis donner une écriture sous la forme 
  $\cfrac{a}{10^{n}}$. Où $a\in\Z$ et $n\in\N$.
  \begin{enumerate}
    \item $a=\cfrac{9}{20}$
    \item $b=\cfrac{7}{50}$
    \item $c=\cfrac{13}{4}$
    \item $d=\cfrac{17}{5}$
  \end{enumerate}
\end{exo}

\begin{exo}
  Ecrire les nombres suivants sous forme décimale.
  \begin{enumerate}
    \item $a=\cfrac{11}{20}$
    \item $b=5+\cfrac{3}{10}+\cfrac{9}{100}+\cfrac{2}{1000}$
    \item $c=\cfrac{2}{100}-\cfrac{5}{10}$
    \item $d=\cfrac{-7}{40}$
  \end{enumerate}
\end{exo}

\begin{exo}
Le nombre $4,156$ peut s'écrire sous la forme: $4,156=4+\cfrac{1}{10}+\cfrac{5}{100}+\cfrac{6}{1000}$. Faire de même avec les nombres suivants:
\begin{enumerate}
\item $A=5,432$.
\item $B=0,45$
\item $C=0,867$
\item $D=-7,1324$	
\end{enumerate}
	
\end{exo}

\newpage

\begin{exo}
.
\begin{enumerate}
\item Donner la valeur décimale des fractions suivantes: $\cfrac{1}{2}$; $\cfrac{1}{4}$; $\cfrac{1}{5}$; $\cfrac{1}{8}$ et $\cfrac{1}{10}$.
\item En déduire la valeur décimale des fractions suivantes:
 $\cfrac{5}{2}$; $\cfrac{3}{5}$; $\cfrac{7}{10}$; $\cfrac{7}{4}$ et $\cfrac{9}{5}$. 
 \item Donner la valeur décimale de: $\cfrac{7}{80}$ et $\cfrac{9}{32}$. 	
\end{enumerate}
	
\end{exo}

\begin{exo}
Mettre les fractions suivantes au même dénominateur: $\cfrac{1}{2}$; $\cfrac{1}{3}$; $\cfrac{1}{4}$ et $\cfrac{1}{5}$. 	En déduire la valeur de $\cfrac{1}{2}+\cfrac{1}{3}+\cfrac{1}{4}+\cfrac{1}{5}$.
\end{exo}


\begin{exo}
Calculer les expressions suivantes et donner le résultat sous la forme d'une fraction irréductible.
\begin{multicols}{5}\noindent
$$A= \cfrac{-72}{5}+\cfrac{96}{35}\times\cfrac{-5}{12}\quad$$
\columnbreak
$$B = \cfrac{\cfrac{-8}{3}-6}{\cfrac{-2}{3}+4}\quad$$
\columnbreak
$$ C= \frac{-1}{2}\div\left(\frac{3}{2}+\frac{13}{7}\right)\quad$$
\columnbreak
$$D= \cfrac{\cfrac{8}{3}+10}{\cfrac{-1}{8}+10} $$
\columnbreak
$$E= \cfrac{7}{4}\div\left(\cfrac{-9}{10}+\cfrac{-4}{7}\right)$$
\end{multicols}

\end{exo}
	
\begin{exo}
Ecrire $10^{n}$ pour $n\in\{-5;-4;-3;-2;-1;0;1;2;3;4;5\}$. On rappelle que si $n\in\N$ on a $10^{-n}=\frac{1}{10^{n}}$. 	
\end{exo}


\begin{exo}
    Complétez les pointillés par un des symboles $\in$, $\notin$, $\subset$, $\not\subset$
    \begin{multicols}{3}
        \begin{enumerate}
            \item $-4\;\ldots\;\mathbb{N}$
            \item $3\;\ldots\;\mathbb{Z}$
            \item $\mathbb{N}\;\ldots\;\mathbb{Z}$
            \item $\mathbb{D}\;\ldots\;\mathbb{Z}$
            \item $\frac{3}{16}\;\ldots\;\mathbb{D}$
            \item $\frac{16}{3}\;\ldots\;\mathbb{D}$
            
        \end{enumerate}
    \end{multicols}
    \end{exo}

\newpage

\setcounter{exo}{11}


\begin{exo}
    \begin{enumerate}
        \item Exprimer sous la forme d'intervalle, d'inégalité et de représentation graphique l'ensemble des nombres $x$ qui vérifient $\lvert x \rvert \leq 3$
        \item Généraliser en exprimant d'une de ces trois manières l'ensemble des $x$ qui vérifient $\lvert x \rvert \leq \ell$ pour un nombre réel $\ell > 0$ donné.
        \item En utilisant la définition de la distance entre deux nombres, Exprimer l'ensemble des nombres qui sont à une distance de moins de $\ell$ du nombre $a$  pour $a \in \R$ et $\ell \in \R^+$
    \end{enumerate}

    
\end{exo}

\begin{exo}
  Placez les nombres décimaux suivants sur la droite des réels, en utilisant l'échelle 1 unité = 8 centimètres, puis écrivez les nombres dans l'ordre croissant séparés par des $<$ :
  $$\frac{7}{20}; 0,5; 0,03145; \frac{39}{40}$$
\end{exo}

\begin{exo}
  On donne des nombres réels. Donner un encadrement des nombres entre deux 
  décimaux à $10^{-2} = 0{,}01$ près. Puis donner le résultats sous forme d'un intervalle d'amplitude $10^{-2}$.
  Exemple: $\pi=3{,}1415235......$ alors $3,14<\pi<3,15\Longleftrightarrow 
  \pi\in]3{,}14;3{,}15[$.
 \begin{enumerate}
   \item $a=\sqrt{3}$
   \item $b=-\frac{\pi}{5}$
   \item $c=53{,}2109$
   \item $d=\frac{8}{7}$
 \end{enumerate}
\end{exo}

\begin{exo}
  Donner traduire les encadrements ou les inégalités suivants sous forme 
  d'intervalles. 
  Exemples:  ou 
  
  \begin{enumerate}
  \item Exemple 1 : $-2,56<x\leq 4,7\Longleftrightarrow x\in ]-2,56;4,7]$
  \item Exemple 2 : $x\geq 7\Longleftrightarrow x\in[7;+\infty[$.
    \item $-2\leq x<-1,6 $
    \item $x\leq -7,2$
    \item $-1\leq a\leq 1$
    \item $-7,56<y$
    \item $-9,29<z\leq 9$
  \end{enumerate}
\end{exo}

\begin{exo}
  Représenter graphiquement chaque intervalle suivant sur une droite réelle.
  \begin{enumerate}
      \item $[3;6]$
      \item $[-2;4[$
      \item $]-\infty;5]$
  \end{enumerate}
\end{exo}

\begin{exo}
  Pour chaque description des intervalles A et B, dire si les ensembles $A \cap B$ et $A \cup B$ sont des intervalles et si oui, écrire cet intervalle.
  \begin{enumerate}
      \item $A=[1;4]$ et $B=[2;6]$
      \item $A=[2;5]$ et $B=[-2;8]$
      \item $A=[1;2]$ et $B=[3;4]$
  \end{enumerate}
\end{exo}

\begin{exo}
    Recopiez et complétez le tableau suivant en faisant correspondre intervalle, inégalités, et représentation graphique
    \begin{center}
    \renewcommand{\arraystretch}{2}
    \begin{tabular}{|c|c|c|}
    \hline
    Intervalle	& C'est l'ensemble des réels $x$ tels que &	Représentation graphique sur la droite des réels\\
    \hline
    $]-2 ; 6[$ &  & \\
    \hline
     & $x < 7$ & \\
    \hline
     &  & \begin{tikzpicture}[>=latex]
    \draw[->] (0,0) --(6,0);
    \draw[blue, line width = 3pt] (2,0) --(5,0);
    \node[blue, line width = 3pt] at (5,0) {$\Big[$};
    \node[blue, line width = 3pt] at (2,0) {$\Big[$};
    \node[below=8pt] at (5,0) {$5$};
    \node[below=8pt] at (2,0) {$2$};
    %\foreach \xp in {2.2,2.4,...,4.8}{\node[] at (\xp,0) {/};}
\end{tikzpicture}\\
    \hline
    $[16 ; +\infty[$ & & \\
    \hline 
    & $6 \leq x \leq 8$  & \\
    \hline
    \end{tabular}
    \end{center}
\end{exo}

\end{document}