\documentclass[11pt,a4paper]{exam}

\usepackage[utf8]{inputenc}
\usepackage{sansmathfonts}
\usepackage[T1]{fontenc}
\renewcommand*\familydefault{\sfdefault} %% Only if the base font of the document is to be sans serif

\usepackage[french]{babel}
%\renewcommand\sfdefault{phv}
\usepackage[left=2cm, right=2cm, bottom=0.5cm, top=0.5cm]{geometry}


\usepackage{fancyhdr}
\usepackage{framed}
\usepackage{multicol}
\usepackage{parskip}
%%%%%%%%%%%%%%%%%%%%%%%%%%%%%%%%%%%%%%%%%%%%%%%%%%

%\newcommand{\TDoc}[1]{
%\begin{center}
%{\setlength{\fboxsep}{10pt}  % Ecart texte-boite
%\shadowbox{\textbf{\Large{#1}}}}
%\end{center}
%\vspace{1.5cm}}
    
 \usepackage{fancybox}   %pour l'encadré du titre shadowbox
 \usepackage{stmaryrd}   %pour utiliser correctement les crochets pour les ensembles de définitions
 \usepackage[normalem]{ulem}
 \usepackage{graphicx}
 %\usepackage{wrapfig} %texte coulé autour d'une image
 \usepackage{soul} % souligné
 \usepackage{nonfloat}
 \usepackage[standard]{ntheorem}
 \usepackage{array}
 \usepackage{arydshln}
 \usepackage{graphicx}
%MATHEMATIQUES
\usepackage{amsmath,amsfonts} 

\usepackage{tikz}
\usetikzlibrary{arrows.meta}
\newcommand{\N}{\mathbb{N}}
\newcommand{\Z}{\mathbb{Z}}
\newcommand{\D}{\mathbb{D}}
\newcommand{\Q}{\mathbb{Q}}
\newcommand{\R}{\mathbb{R}}
\newcommand{\Co}{\mathbb{C}}
\newcommand{\K}{\mathbb{K}}
\newcommand{\F}{\mathbb{F}}


%%%%%%%%%%%%%%%%%%%%%%%%%%%%%%%%%%%

\makeatletter
%%%%%%%%%%%%%%%%%%% debut fichier boiboites.sty %%%%%%%%%%%%%%%%%%%%%%
\RequirePackage{xkeyval}
\RequirePackage{tikz}
\usetikzlibrary{intersections}
\usetikzlibrary{positioning}
\usetikzlibrary{3d}
\RequirePackage{amssymb}

\define@key{boxedtheorem}{titlecolor}{\def\titlecolor{#1}}
\define@key{boxedtheorem}{titlebackground}{\def\titlebackground{#1}}
\define@key{boxedtheorem}{background}{\def\background{#1}}
\define@key{boxedtheorem}{titleboxcolor}{\def\titleboxcolor{#1}}
\define@key{boxedtheorem}{boxcolor}{\def\boxcolor{#1}}
\define@key{boxedtheorem}{thcounter}{\def\thcounter{#1}}
\define@key{boxedtheorem}{size}{\def\size{#1}}
\presetkeys{boxedtheorem}{titlecolor = black, titlebackground = white, background = white,%
                         titleboxcolor = black, boxcolor = black, thcounter=, size = .9\textwidth}{}

\newcommand{\couleurs}[1][]{%
    \setkeys{boxedtheorem}{#1}
    \tikzstyle{fancytitle} =[draw=\titleboxcolor, rounded corners, fill=\titlebackground,
                            text= \titlecolor]
    \tikzstyle{mybox} = [draw=\boxcolor, fill=\background, very thick,
                        rectangle, rounded corners, inner sep=10pt, inner ysep=20pt]
}


\newtheorem{exo}{Exercice}
%%%%%%%%%%%%%%%%%%%%%%%%%%%%%
\newlength{\longA}
\newlength{\longB}
\newenvironment{BoiteShadow}[3][\linewidth]{%
\addtolength{\longA}{#2}
\addtolength{\longB}{#3}
\begin{Sbox}\begin{minipage}{#1}}%
{\end{minipage}\end{Sbox}%
\setlength{\fboxsep}{\longA}
\setlength{\shadowsize}{\longB}
\shadowbox{\unhbox\@Sbox}\par}
\makeatother
%\author{Augustin WENGER}

\date{}

%%%% fin du préambule, on passe au contenu : tout le texte entre
%%%% \begin{document} et \end{document} 

\begin{document}

\begin{minipage}{0.15\textwidth}
    Nom : 
\end{minipage}
\begin{minipage}{0.8\textwidth}
    \begin{flushright}
        Contrôle - TSTMG2 - 21 octobre 2022
    \end{flushright}
\end{minipage}
\vspace{3pt}
\hline
\vspace{3pt}
\fbox{\parbox{\dimexpr\linewidth-2\fboxsep-2\fboxrule}{Dans chaque exercice, il est important de faire figurer vos étapes de calcul et pas seulement la réponse. Le soin apporté à la rédaction et à la présentation sera pris en compte dans la notation.}}

\vspace{8pt}

\begin{exo}
    Répondez aux questions en cochant la case correspondante (Plusieurs réponses peuvent être correctes): 
    \begin{enumerate}
        \item Une suite \textbf{arithmétique} est telle que $u_3=10$ et $u_{12}=37$. Pour calculer sa raison, je fais le calcul : \newline
        \begin{oneparcheckboxes}
            \choice $\frac{12-3}{37-10}$
            \choice $\frac{37-10}{12-3+1}$
            \choice $-\frac{10-37}{12-3}$
            \choice $\frac{37-10}{12-3}$
        \end{oneparcheckboxes}
        \item Une classe passe de $25$ à $23$ élèves. Le taux d'évolution du nombre d'élèves est de \newline
        \begin{oneparcheckboxes}
            \choice $-2\%$
            \choice environ $-8{,}7\%$
            \choice $-8\%$
            \choice $+8\%$
        \end{oneparcheckboxes}
        \item Pour calculer le pourcentage de baisse d'un parfum qui vaut initialement $35$€ et baisse de $7$€, je fais le calcul : \newline
        \begin{oneparcheckboxes}
            \choice $\frac{35-7}{35}$
            \choice $\frac{7}{35}$
            \choice $-\frac{7}{35}$
            \choice $-\frac{35-7}{35}$
        \end{oneparcheckboxes}
        \item Une suite \textbf{géométrique} est telle que $u_{4}=9$ et $u_2=1$. Sa raison peut-être de : \newline
        \begin{oneparcheckboxes}
            \choice $\frac{1}{3}$
            \choice $3$
            \choice $-\frac{1}{3}$
            \choice $-3$
        \end{oneparcheckboxes}
        \item Le prix de chaussures baisse de $50\%$ entre 2020 et 2021, puis de $10\%$ entre 2021 et 2022. Au total, le prix a baissé de : \newline
        \begin{oneparcheckboxes}
            \choice $60\%$
            \choice $65\%$
            \choice environ $62{,}3\%$
            \choice $55\%$
        \end{oneparcheckboxes}
    \end{enumerate}
\end{exo}
\vspace{0.5cm}
\begin{exo}Calculez les premières valeurs des suites définies :

\renewcommand{\arraystretch}{2}
\begin{center}
    \begin{tabular}{|c|c|c|c|c|c|}
        \hline
          Définition de $u_n$&$\;u_1\;$&$\;u_2\;$&$\;u_3\;$&$\;u_4\;$&$\;u_5\;$\\
          \hline
         $u_n=n-\frac{3}{n}$ pour $n \geq 1$& & & & &\\
          \hline
         $u_n=3n^2+2$ pour $n \geq 1$& & & & &\\
          \hline
         $u_1=3$ pour $n \geq 1 u_{n+1} = -\frac{1}{u_n}$& & & & &\\
          \hline
         $u_1=6$ et pour $n \geq 1, u_{n+1}=\frac{u_n}{2} - 1$& & & & &\\
          \hline
    \end{tabular}
\end{center}
\end{exo} 
\vspace{0.5cm}
\begin{exo}
    Pour chacune des situations présentées, donner le calcul que l'on doit effectuer pour trouver la réponse demandée, puis calculer cette réponse. 
    \begin{enumerate}
    \item Les émissions de CO$_2$ en France sont passées de $451$ mégatonnes à $441$ mégatonnes entre 2018 et 2019. Calculer le taux d'évolution $A$ des émissions de 2018 à 2019.\newline
    	\fbox{
        	\begin{minipage}[t][1.5cm][t]{0.965\linewidth}
                A=
        	\end{minipage}
        }
    \item Une entreprise augmente son chiffre d'affaires de $10\%$ par an. En 2022, elle a réalisé $264$ M€ de chiffre d'affaires. Quel a été son chiffre d'affaire $B$ en 2021?\newline
    \fbox{
    	\begin{minipage}[t][1.5cm][t]{0.965\linewidth}
            B=
    	\end{minipage}
    }
    \item Le prix d'une action en bourse, qui valait initialement $60$€, baisse de $50\%$ le lundi, puis augmente de $50\%$ le mardi. Quel est son prix $C$ à la fin de la journée du mardi? \newline
    \fbox{
    	\begin{minipage}[t][1.5cm][t]{0.965\linewidth}
            C=
    	\end{minipage}
    }
    \item Le coût du beurre vient d'augmenter de $20\%$. Une entreprise fabriquant des galettes avait des coûts totaux de $4$€ par paquet, et $25\%$ de ces coûts étaient liés à l'achat de beurre. Quel est la nouvelle part $D$ de coûts liés à l'achat du beurre, toutes choses égales par ailleurs?  \newline
    \fbox{
    	\begin{minipage}[t][1.5cm][t]{0.965\linewidth}
            D=
    	\end{minipage}
    }
    \end{enumerate}
\end{exo}
\vspace{1cm}

\begin{exo}
\begin{enumerate}
    \item Une entreprise nommée "Aperture Science" vous propose de vous embaucher pour un salaire annuel de $20000$€ la première année, assorti d'une augmentation de $1000$€ par an.
    \begin{enumerate}
        \item On appelle $u_n$ la suite qui représente votre salaire la n-ième année. Quelle est la nature de la suite $u_n$ ? 
        \item Calculez $u_2$ et $u_3$
        \item Quel salaire recevriez-vous la $15$ème année si vous signez pour cette entreprise?
        \item Calculez le montant total de salaire reçu dans cette entreprise, en supposant que vous restez y travailler pour 15 années complètes.
    \end{enumerate}
    \item L'équipe de recrutement d'une autre entreprise concurrente, "Batibox", vous fait une autre proposition : Un salaire annuel de $18000$€ la première année, et une augmentation annuelle de 6\%
    \begin{enumerate}
        \item On appelle $v_n$ la suite qui représente votre salaire la n-ième année. Quelle est la nature de la suite $v_n$ ? 
        \item Calculez $v_2$ et $v_3$
        \item Quel salaire recevriez-vous la $15$ème année si vous choisissez de signer avec Batibox ?
        \item Calculez le montant total de salaire reçu dans cette entreprise, en supposant que vous restez y travailler pour 15 années complètes.
    \end{enumerate}
    \item Vous pensez vraiment travailler dans la même entreprise pendant les 15 prochaines années. Quel est le meilleur choix parmi les deux possibilités proposées, et pourquoi?  
\end{enumerate}
\end{exo}
\fbox{
	\begin{minipage}[t][18cm][t]{0.965\linewidth}
      ~
	\end{minipage}
}
\newpage
\setcounter{exo}{0}
\begin{minipage}{0.15\textwidth}
    Nom : 
\end{minipage}
\begin{minipage}{0.8\textwidth}
    \begin{flushright}
        Contrôle - TSTMG2 - 21 octobre 2022
    \end{flushright}
\end{minipage}
\vspace{3pt}
\hline
\vspace{3pt}
\fbox{\parbox{\dimexpr\linewidth-2\fboxsep-2\fboxrule}{Dans chaque exercice, il est important de faire figurer vos étapes de calcul et pas seulement la réponse. Le soin apporté à la rédaction et à la présentation sera pris en compte dans la notation.}}

\vspace{8pt}

\begin{exo}
    Répondez aux questions en cochant la case correspondante (Plusieurs réponses peuvent être correctes): 
    \begin{enumerate}
        \item Une classe passe de $23$ à $25$ élèves. Le taux d'évolution du nombre d'élèves est de \newline
        \begin{oneparcheckboxes}
            \choice environ $+8{,}7\%$
            \choice $+8\%$
            \choice $+2\%$
            \choice $-8\%$
        \end{oneparcheckboxes}
        \item Pour calculer le pourcentage de baisse d'un parfum qui vaut initialement $48$€ et baisse de $3$€, je fais le calcul : \newline
        \begin{oneparcheckboxes}
            \choice $\frac{3}{48}$
            \choice $-\frac{3}{48}$
            \choice $\frac{48-3}{48}$
            \choice $-\frac{48-3}{48}$
        \end{oneparcheckboxes}
        \item Une suite \textbf{arithmétique} est telle que $u_4=20$ et $u_{16}=48$. Pour calculer sa raison, je fais le calcul : \newline
        \begin{oneparcheckboxes}
            \choice $\frac{48-20}{16-4+1}$
            \choice $\frac{16-4}{48-20}$
            \choice $\frac{48-20}{16-4}$
            \choice $-\frac{20-48}{16-4}$
        \end{oneparcheckboxes}
        \item Une suite \textbf{géométrique} est telle que $u_{5}=8$ et $u_3=2$. Sa raison peut-être de : \newline
        \begin{oneparcheckboxes}
            \choice $2$
            \choice $-2$
            \choice $\frac{1}{2}$
            \choice $-\frac{1}{2}$
        \end{oneparcheckboxes}
        \item Le prix de chaussures augmente de $50\%$ entre 2020 et 2021, puis de $10\%$ entre 2021 et 2022. Au total, le prix a augmenté de : \newline
        \begin{oneparcheckboxes}
            \choice $60\%$
            \choice $65\%$
            \choice environ $62{,}3\%$
            \choice $55\%$
        \end{oneparcheckboxes}
    \end{enumerate}
\end{exo}
\vspace{0.5cm}
\begin{exo}Calculez les premières valeurs des suites définies :

\renewcommand{\arraystretch}{2}
\begin{center}
    \begin{tabular}{|c|c|c|c|c|c|}
        \hline
          Définition de $u_n$&$\;u_1\;$&$\;u_2\;$&$\;u_3\;$&$\;u_4\;$&$\;u_5\;$\\
          \hline
         $u_n=\frac{6}{n}+n$ pour $n \geq 1$& & & & &\\
          \hline
         $u_n=4n^2-7$ pour $n \geq 1$& & & & &\\
          \hline
         $u_1=2$ et pour $n \geq 1, u_{n+1}=\frac{u_n}{2} + 3$& & & & &\\
          \hline
         $u_1=6$ pour $n \geq 1 u_{n+1} = -\frac{1}{u_n}$& & & & &\\
          \hline
    \end{tabular}
\end{center}
\end{exo} 
\vspace{0.5cm}
\begin{exo}
    Pour chacune des situations présentées, donner le calcul que l'on doit effectuer pour trouver la réponse demandée, puis calculer cette réponse. 
    \begin{enumerate}
    \item Les émissions de CO$_2$ en France sont passées de $441$ mégatonnes à $399$ mégatonnes entre 2019 et 2020. Calculer le taux d'évolution $A$ des émissions de 2019 à 2020.\newline
    	\fbox{
        	\begin{minipage}[t][1.5cm][t]{0.965\linewidth}
                A=
        	\end{minipage}
        }
    \item Une entreprise augmente son chiffre d'affaires de $10\%$ par an. En 2022, elle a réalisé $572$ M€ de chiffre d'affaires. Quel a été son chiffre d'affaire $B$ en 2021?\newline
    \fbox{
    	\begin{minipage}[t][1.5cm][t]{0.965\linewidth}
            B=
    	\end{minipage}
    }
    \item Le prix d'une action en bourse, qui valait initialement $80$€, augmente de $50\%$ le lundi, puis baisse de $50\%$ le mardi. Quel est son prix $C$ à la fin de la journée du mardi? \newline
    \fbox{
    	\begin{minipage}[t][1.5cm][t]{0.965\linewidth}
            C=
    	\end{minipage}
    }
    \item Le coût du beurre vient d'augmenter de $30\%$. Une entreprise fabriquant des galettes avait des coûts totaux de $3$€ par paquet, et $20\%$ de ces coûts étaient liés à l'achat de beurre. Quel est la nouvelle part $D$ de coûts liés à l'achat du beurre, toutes choses égales par ailleurs?  \newline
    \fbox{
    	\begin{minipage}[t][1.5cm][t]{0.965\linewidth}
            D=
    	\end{minipage}
    }
    \end{enumerate}
\end{exo}
\vspace{1cm}

\begin{exo}
\begin{enumerate}
    \item Une entreprise nommée "Aperture Science" vous propose de vous embaucher pour un salaire annuel de $22000$€ la première année, assorti d'une augmentation de $1000$€ par an.
    \begin{enumerate}
        \item On appelle $u_n$ la suite qui représente votre salaire la n-ième année. Quelle est la nature de la suite $u_n$ ? 
        \item Calculez $u_2$ et $u_3$
        \item Quel salaire recevriez-vous la $15$ème année si vous signez pour cette entreprise?
        \item Calculez le montant total de salaire reçu dans cette entreprise, en supposant que vous restez y travailler pour 15 années complètes.
    \end{enumerate}
    \item L'équipe de recrutement d'une autre entreprise concurrente, "Batibox", vous fait une autre proposition : Un salaire annuel de $20000$€ la première année, et une augmentation annuelle de 5\%
    \begin{enumerate}
        \item On appelle $v_n$ la suite qui représente votre salaire la n-ième année. Quelle est la nature de la suite $v_n$ ? 
        \item Calculez $v_2$ et $v_3$
        \item Quel salaire recevriez-vous la $15$ème année si vous choisissez de signer avec Batibox ?
        \item Calculez le montant total de salaire reçu dans cette entreprise, en supposant que vous restez y travailler pour 15 années complètes.
    \end{enumerate}
    \item Vous pensez vraiment travailler dans la même entreprise pendant les 15 prochaines années. Quel est le meilleur choix parmi les deux possibilités proposées, et pourquoi?  
\end{enumerate}
\end{exo}
\fbox{
	\begin{minipage}[t][18cm][t]{0.965\linewidth}
      ~
	\end{minipage}
}
\end{document}

